\documentclass{article}
\usepackage[utf8]{inputenc}
\usepackage{titlesec}
\usepackage[a4paper, total={7in, 10.7in}]{geometry}
\usepackage{hyperref}
\usepackage{xcolor}
\usepackage{enumitem}

\hypersetup{
    colorlinks=true,
    linkcolor=cyan,
    filecolor=magenta,      
    urlcolor=[HTML]{0073b1},
}

\setcounter{secnumdepth}{0}
\setlength{\parindent}{0em}
\titlespacing*{\section}{0pt}{1em}{0em}

\begin{document}

\pagenumbering{gobble}

\begin{center}
  {\Large\textbf{Zete Dai}} \\
  418 60th St, Brooklyn, NY 11220 $|$
  347-295-7108 $|$
  \href{mailto:zd790@nyu.edu}{zd790@nyu.edu} \\
  \href{https://www.linkedin.com/in/zete-dai-567077153/}{linkedin.com/in/zete-dai-567077153} $|$
  \href{https://github.com/SiAce}{github.com/SiAce} $|$
  \href{https://www.siacespark.com}{siacespark.com}
  
\end{center}

\section{Education}
\hrule
\vspace{1em}

\textbf{New York University}, New York, United States \hfill
Sep 2019 - May 2021 \\
Master of Science in Computer Science \\
GPA: 3.39

\vspace{1em}

\textbf{Shanghai Jiao Tong University}, Shanghai, China \hfill
Sep 2015 - Jun 2019 \\
Bachelor of Science in Physics \\
Bachelor of Engineering in Computer Technology and Application \\
GPA: 3.47

\section{Experience}
\hrule
\vspace{1em}

\textbf{Research Assistant}: Physics Computational Team, Shanghai Jiao Tong University \hfill
Jan 2016 - Jun 2019

\begin{itemize}[nosep]
	\item Found the highest possible pressure on theory by using COMSOL to simulate double-stage diamond anvil cell
	using finite element analysis.
	\item Used Python to find the ideal shape by data profiling, cleansing, analyzing, modeling, and visualizing.
	\item Developed a command-line Python program to convert between the index of crystal orientation and plane.
\end{itemize}

\vspace{1em}

\textbf{Research Assistant}: PandaX Team, Shanghai Jiao Tong University \hfill
Jan 2017 - Jun 2017

\begin{itemize}[nosep]
	\item Improved the efficiency of writing and processing the data generated by dark-matter-detecting device using
	MongoDB (C++) and Python.
\end{itemize}

\section{Skills}
\hrule
\vspace{1em}

\textbf{Overview}: Machine Learning, Deep Learning, Reinforcement Learning, Data Analysis, Web Development,
iOS/Android Development, Game Development, Continuous Integration

\textbf{Frameworks}: TensorFlow, PyTorch, scikit-learn, Pandas, Hadoop, Spark, Flask, Django, Express.js, React

\textbf{Languages}: Python, C++, Java, Rust, C\texttt{\#}, R, HTML/CSS, JavaScript

\textbf{Databases}: PostgreSQL, SQLite, MongoDB, Redis

\textbf{Softwares}: Jira, Unreal Engine, Unity, Docker, Kubernetes, Kafka

\textbf{Services}: Google Cloud, Microsoft Azure, Google Kubernetes Engine, Travis CI, GitHub Actions, REST API

\section{Honors and Awards}
\hrule
\vspace{1em}

\$8000 Grad School of Engineering Scholarship, New York University \hfill
May 2019

Third-Class Scholarship, Shanghai Jiao Tong University \hfill
Dec 2016

\section{Projects}
\hrule
\vspace{1em}

\textbf{Store Management System} (Python, Flask, Flask-RESTful, Flask-SQLAlchemy, SQLite)

\begin{itemize}[nosep]
	\item Built a backend RESTful API to query and manage stores, items, and users based on Flask and Flask-RESTful.
	\item Stored the stores, items and users data in the SQLite database, and used Flask-SQLAlchemy to build Object–Relational Mapping (ORM) between SQLite records and Python objects.
\end{itemize}

\vspace{1em}

\textbf{Computer Vision Application} (C++, Qt)

\begin{itemize}[nosep]
	\item Created a GUI that can display the coordinates and RGB values of the point hovered by mouse in real-time.
	\item Implemented color image processing, binarization, algebra and geometry operations, contrast adjustment.
	\item CV operations include smoothing filter, edge detection, binary morphology,
	grayscale morphology.
\end{itemize}

\vspace{1em}

\textbf{YelpCamp} (JavaScript, Node.js, Express.js, MongoDB, Mongoose, Bootstrap)

\begin{itemize}[nosep]
	\item Created a full-stack website using Express.js (Node.js) that can let users register, log in, log out. Users can also view, add, edit, delete
	campsites, and reviews for campsites.
	\item Deployed the website at Heroku.
\end{itemize}

\vspace{1em}

\textbf{Neuroevolution of Self-Interpretable (NSI) Agents} (Python, TensorFlow, Procgen)

\begin{itemize}[nosep]
	\item Applied NSI using LSTM and self-attention to build a general video game agent on three Procgen game environments: Coinrun, BigFish, Plunder.
	\item Applied Double Deep Q Neural Network using TensorFlow to build a reinforcement agent for each game.
\end{itemize}

\end{document}
